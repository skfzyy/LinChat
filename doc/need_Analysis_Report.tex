\documentclass[UTF8]{ctexart}
\title{LinChat 需求文档}
\author{沈航可}
\date{2021年8月14日}
\pagestyle{plain}
\usepackage[a4paper, left = 2cm, right = 2cm, bottom=2cm,top=2cm]{geometry}

% \usepackage[UTF8,scheme=plain]{ctex}
\setCJKmainfont{SimSun}

\usepackage{fontspec}
\setmainfont{Times New Roman}

\begin{document}
\maketitle
\section{项目说明}
\subsection{立项目的}
微信作为日常的大众通讯工具,目前仅开发了Windows和MacOS版本,对于部分将Linux作为工作平台的
程序员和其他领域的工程师来说,Linux平台缺乏一款通用的通讯工具.虽然目前可以通过Wine将微信引入
Linux平台,但此工具仍然还有一些缺陷.例如缺乏对原生Linux的支持、文件系统配置仍然还是沿用了Window的习惯
等,这降低了工具的易用性和通用性.
同时,作为一个喜欢科技化界面的程序员来说,微信的UI仍然也无法让人特别满意.

因此,本项目旨在开发一款能够在多平台(目前主要在Linux系统)运行的通用聊天工具,通过编写富有科技化的用户界面,
使聊天成为一件酷炫的事情.

\section{拟使用的开发框架与工具}
针对本项目的需求,选定的框架和技术方案需要满足以下几个需求:

1.高效.本项目需要能够高效地运行,尽可能的降低网络延迟的同时,能够在配置尽可能低的平台上运行.

2.跨平台.聊天工具不应仅限于运行在Linux端,对于Windows、macOS、Android、IOS等平台的兼容性也至关重要.目前开发的版本
主要实现对于PC平台(Win,Linux,macOS)的支持.

3.界面.具有一定的科技感.初始的界面皮肤倾向于使用类似项目:https://github.com/seenaburns/dex-ui与项目https://github.com/GitSquared/edex-ui.
其中, https://github.com/seenaburns/dex-ui 使用c++开发,更具参考性.

\subsection{UI框架}
由于QT具有良好的跨平台性,通过开源对个人开发者开放,同时,QT对OpenGL也有良好的支持,因此拟选定QT作为编写UI
的基础框架.

\end{document}